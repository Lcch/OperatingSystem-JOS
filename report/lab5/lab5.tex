\documentclass[GBK,winfonts,a4paper,10pt]{ctexart}
\usepackage{fancyhdr}
\usepackage{indentfirst}
\usepackage{graphics}
\usepackage{enumerate}
\usepackage{framed}
\usepackage{amsmath}
\usepackage{graphicx}
\usepackage{setspace}
\usepackage{hyperref}
\usepackage{mdwlist}
\usepackage{algorithm}
\usepackage{algorithmic}
\usepackage{listings}
\usepackage{xcolor}
\usepackage{marvosym,listings,etoolbox}
\usepackage{geometry}

\lstset{numbers=left, numberstyle=\small, keywordstyle=\color{blue!70}, commentstyle=\color{red!50!green!50!blue!50}, frame=shadowbox, rulesepcolor=\color{red!20!green!20!blue!20},escapeinside=``, xleftmargin=2em,xrightmargin=2em, aboveskip=1em, literate={@}{\MVAt}1}

\patchcmd{\verb}{\dospecials}{\dospecials\atspecial}{}{}
\def\atspecial{\begingroup\lccode`~=`@
  \lowercase{\endgroup\let~}\MVAt
  \catcode`@=\active}
  


\newcommand{\tabincell}[2]{\begin{tabular}{@{}#1@{}}#2\end{tabular}}%
       
\lstdefinestyle{customc}{
  belowcaptionskip=1\baselineskip,
  breaklines=true,
  frame=single,
  xleftmargin=\parindent,
  language=C,
  showstringspaces=false,
  basicstyle=\fontsize{8pt}{8pt}\ttfamily,
  keywordstyle=\bfseries\color{green!40!black},
  commentstyle=\itshape\color{purple!40!black},
  identifierstyle=\color{blue},
  stringstyle=\color{orange},
  tabsize=4,
  numbers=none,
  mathescape=false,
}

\lstset{escapechar=@,style=customc}

\pagestyle{fancy}
\hypersetup{pdfborder=0 0 0}

\usepackage{clrscode}

\usepackage{latexsym}

\begin{document}

\rhead{}
\lhead{}
\cfoot{\thepage}
\renewcommand{\footrulewidth}{0.4pt}
%\renewcommand{\thesection}{}
\renewcommand{\algorithmicrequire}{\textbf{Input:}}
\renewcommand{\algorithmicensure}{\textbf{Output:}}
\setlength{\tabcolsep}{2pt}

\setlength{\parindent}{2em}

\thispagestyle{fancy}


\title{Operating System MIT 6.828 JOS Lab5 Report}
\author{Computer Science \\ ChenHao(1100012776) }
\date{\today}
\maketitle

\thispagestyle{fancy}

\tableofcontents

\newpage

\begin{section}{ Exercise 1 }
\par
仅给file system environment I/O权限,如果同时有多个environment享有I/O权限,则会对于中断的分配到对应的user-mode environment造成很大困扰。
\par
IOPL有4种特权级,其中0级的特权最高,3级最低,而此处是给予用户进程权限,因此给予FL_IOPL_3。
\begin{lstlisting}[language=C]
    if (type == ENV_TYPE_FS)
        e->env_tf.tf_eflags |= FL_IOPL_3;        
\end{lstlisting}
\end{section}

\begin{section}{ Question }
\par
e->env_tf会在产生trap时,由硬件以及中断处理程序进行保存,在env_pop_tf()中恢复。
\end{section}

\begin{section}{ Exercise 2 }
\par
本质上这就是一个page fault handler,不同之处在于拷贝信息,一个是从内存中,这个是从硬盘中。
\begin{lstlisting}[language=C]
    	addr = ROUNDDOWN(addr, PGSIZE);
	r = sys_page_alloc(0, addr, PTE_W | PTE_U | PTE_P);
	if (r < 0) panic("bc_pgfault sys_page_alloc error : %e\n", r);

	r = ide_read(blockno * BLKSECTS, addr, BLKSECTS);
	if (r < 0) panic("bc_pgfault ide_read error : %e\n", r);       
\end{lstlisting}
\end{section}

\begin{section}{ Challenge 1 }
\end{section}

\begin{section}{ Exercise 3 }
\end{section}

\end{document}



















