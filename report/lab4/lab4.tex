\documentclass[GBK,winfonts,a4paper,10pt]{ctexart}
\usepackage{fancyhdr}
\usepackage{indentfirst}
\usepackage{graphics}
\usepackage{enumerate}
\usepackage{framed}
\usepackage{amsmath}
\usepackage{graphicx}
\usepackage{setspace}
\usepackage{hyperref}
\usepackage{mdwlist}
\usepackage{algorithm}
\usepackage{algorithmic}
\usepackage{listings}
\usepackage{xcolor}
\usepackage{marvosym,listings,etoolbox}
\usepackage{geometry}

\lstset{numbers=left, numberstyle=\small, keywordstyle=\color{blue!70}, commentstyle=\color{red!50!green!50!blue!50}, frame=shadowbox, rulesepcolor=\color{red!20!green!20!blue!20},escapeinside=``, xleftmargin=2em,xrightmargin=2em, aboveskip=1em, literate={@}{\MVAt}1}

\patchcmd{\verb}{\dospecials}{\dospecials\atspecial}{}{}
\def\atspecial{\begingroup\lccode`~=`@
  \lowercase{\endgroup\let~}\MVAt
  \catcode`@=\active}
  


\newcommand{\tabincell}[2]{\begin{tabular}{@{}#1@{}}#2\end{tabular}}%
       
\lstdefinestyle{customc}{
  belowcaptionskip=1\baselineskip,
  breaklines=true,
  frame=single,
  xleftmargin=\parindent,
  language=C,
  showstringspaces=false,
  basicstyle=\fontsize{8pt}{8pt}\ttfamily,
  keywordstyle=\bfseries\color{green!40!black},
  commentstyle=\itshape\color{purple!40!black},
  identifierstyle=\color{blue},
  stringstyle=\color{orange},
  tabsize=4,
  numbers=none,
  mathescape=false,
}

\lstset{escapechar=@,style=customc}

\pagestyle{fancy}
\hypersetup{pdfborder=0 0 0}

\usepackage{clrscode}

\usepackage{latexsym}

\begin{document}

\rhead{}
\lhead{}
\cfoot{\thepage}
\renewcommand{\footrulewidth}{0.4pt}
%\renewcommand{\thesection}{}
\renewcommand{\algorithmicrequire}{\textbf{Input:}}
\renewcommand{\algorithmicensure}{\textbf{Output:}}
\setlength{\tabcolsep}{2pt}

\setlength{\parindent}{2em}

\thispagestyle{fancy}


\title{Operating System MIT 6.828 JOS Lab4 Report}
\author{Computer Science \\ ChenHao(1100012776) }
\date{\today}
\maketitle

\thispagestyle{fancy}

\tableofcontents

\newpage

\begin{section}{Part A: Multiprocessor Support and Cooperative Multitasking}
\begin{subsection}{Exercise 1}
\par
这是是要我们实现一个用于简单分配内存空间进行映射的函数,利用以前写的boot\_map\_region就可以很容易的实现。
\begin{lstlisting}[language=C]
void *
mmio_map_region(physaddr_t pa, size_t size)
{
	static uintptr_t base = MMIOBASE;

	pa = ROUNDDOWN(pa, PGSIZE);
	boot_map_region(kern_pgdir, base, ROUNDUP(size, PGSIZE), pa, PTE_PCD | PTE_PWT | PTE_W);
	uintptr_t tmp_base = base;
	base += ROUNDUP(size, PGSIZE);
	return (void *) tmp_base;
}
\end{lstlisting}
\end{subsection}

\begin{subsection}{Exercise 2}
\par
这一段是在说明AP是如何启动的,BSP在启动APs前,会先收集CPU信息,APIC IDs, LAPIC的MMIO地址等信息,然后执行boot\_aps,在这段中BSP依次让每个AP启动执行mpentry.S中的代码(BSP将代码复制至内存中MPENTRY\_ADDRESS的位置,并将每个的AP的CS:IP设置好),通过发送STARTUP IPI两次来启动AP(在lapic\_startup函数中说明了,这个启动方式是硬件支持的)之后AP就执行启动程序,等执行完后会给BSP发送CPU\_STARTED的信号。表示结束,BSP在APs的启动过程中一直处于空循环的状态。
\par
因此MPENTRY\_ADDRESS所在的内存地址是不能够被用于分配其它代码或者数据的,因此在page\_init的需要进行修改。
\begin{lstlisting}[language=C]
void
page_init(void)
{
    page_free_list = NULL;
    size_t i;
	size_t nf_lb = IOPHYSMEM / PGSIZE;
    size_t nf_ub = PADDR(boot_alloc(0)) / PGSIZE;
    size_t mpentry_page = MPENTRY_PADDR / PGSIZE;
    for (i = 0; i < npages; i++) {
        if (i != 0 && (i < nf_lb || i >= nf_ub) && i != mpentry_page) {
		    pages[i].pp_ref = 0;
            pages[i].pp_link = page_free_list;
		    page_free_list = &pages[i];
        } else {
            pages[i].pp_ref = 1;
            pages[i].pp_link = NULL;
        }
	}
}
\end{lstlisting}
\end{subsection}

\begin{subsection}{Question 1}
\par
kern/mpentry.S的链接地址在KERNBASE以上,而AP在是在实模式下无法使用页寻址,因此我们需要手动地计算出其所在的物理地址。如果注释了这句,则会产生缺页异常。
\end{subsection}

\begin{subsection}{Exercise 3}
\par
问题:为什么给kern\_pgdir即可,不用每个cpu的pgdir都赋予值?
\begin{lstlisting}[language=C]
static void
mem_init_mp(void)
{    
    int cpu_id;
    for (cpu_id = 0; cpu_id < NCPU; cpu_id++) {
        boot_map_region(kern_pgdir,
                        KSTACKTOP - cpu_id * (KSTKSIZE + KSTKGAP) - KSTKSIZE,
                        KSTKSIZE,
                        PADDR(percpu_kstacks[cpu_id]),
                        PTE_W);
    }
}
\end{lstlisting}
\end{subsection}

\begin{subsection}{Exercise 4}
\par
根据cpu\_id放置在对应的位置即可。
\begin{lstlisting}[language=C]
void
trap_init_percpu(void)
{
	// LAB 4: Your code here:
    int cpu_id = thiscpu->cpu_id;
    thiscpu->cpu_ts.ts_esp0 = KSTACKTOP - cpu_id * (KSTKSIZE + KSTKGAP);
    thiscpu->cpu_ts.ts_ss0 = GD_KD;

    gdt[(GD_TSS0 >> 3) + cpu_id] = SEG16(STS_T32A, (uint32_t) (&(thiscpu->cpu_ts)),
    							sizeof(struct Taskstate), 0);
    gdt[(GD_TSS0 >> 3) + cpu_id].sd_s = 0;

    ltr(GD_TSS0 + (cpu_id << 3));

    lidt(&idt_pd);
}
\end{lstlisting}
\end{subsection}

\begin{subsection}{Exercise 5}
\par
根据题目要求在对应的地方放上lock或者unlock即可,连位置都有注释提示,非常简单。
\end{subsection}

\begin{subsection}{Question 2}
\par
共享内核栈是会存在问题的,在发生中断的时候,对于现场保护的是发生在big kernel lock之前的,因此当共享内核栈很可能会产生参数的混乱。
\end{subsection}

\begin{subsection}{Exercise 6}
\par
这一部分要我们实现Round-Robin Scheduling,即类似循环链表,每次从刚执行的env之后选择最近的一个RUNNABLE的进行执行。代码如下:
\begin{lstlisting}[language=C]
void
sched_yield(void)
{
	struct Env *idle;

	// LAB 4: Your code here.
	int now_env, i;
	if (curenv) {			// thiscpu->cpu_env
		now_env = (ENVX(curenv->env_id) + 1) % NENV;
	} else {
		now_env = 0;
	}
	for (i = 0; i < NENV; i++, now_env = (now_env + 1) % NENV) {
		if (envs[now_env].env_status == ENV_RUNNABLE) {
			//cprintf("I am CPU %d , I am in sched yield, I find ENV %d\n", thiscpu->cpu_id, now_env);
			env_run(&envs[now_env]);
		}
	}
	if (curenv && curenv->env_status == ENV_RUNNING) {
		env_run(curenv);
	}
	// sched_halt never returns
	sched_halt();
}
\end{lstlisting}
\end{subsection}

\begin{subsection}{Question 3-4}
\par
Question 3: 由于Env在mem\_init中分配了内存进行存储并建立了映射,而这个映射在每个CPU上都是一致的,因此lcr3()后对于Env的还是存在虚拟内存到物理内存的映射的。
\par
Question 4: 保存现场是为了之后CPU进行继续处理这个environment的时候保证不会造成错误,
\end{subsection}


\begin{subsection}{Question 7}
\begin{subsubsection}{sys\_exofork}
\par
这个部分是生成一个新的environment,使得其寄存器的值与当前的environment的一样,对于新的environment返回值为0(\%eax=0)。对于该函数返回新的environment的pid号。
\begin{lstlisting}[language=C]
static envid_t
sys_exofork(void)
{
	struct Env * new_env;
	int r = env_alloc(&new_env, curenv->env_id);
	if (r < 0) return r;
	
	new_env->env_status = ENV_NOT_RUNNABLE;
	memcpy((void *)(&new_env->env_tf), (void*)(&curenv->env_tf), sizeof(struct Trapframe));
	
	// for children environment, return 0
	new_env->env_tf.tf_regs.reg_eax = 0;

	return new_env->env_id;
}
\end{lstlisting}
\end{subsubsection}

\begin{subsubsection}{sys\_env\_set\_status}
\par
这个是设置environment的status,按照注释做即可。
\begin{lstlisting}[language=C]
static int
sys_env_set_status(envid_t envid, int status)
{
	if (status != ENV_RUNNABLE && status != ENV_NOT_RUNNABLE) 
		return -E_INVAL;

	struct Env * env;
	int r = envid2env(envid, &env, 1);
	if (r < 0) return r;
	env->env_status = status;
	return 0;
}
\end{lstlisting}
\end{subsubsection}

\begin{subsubsection}{sys\_page\_alloc}
\par
这个函数是为environment创建虚拟地址va的映射页,按照注释一条一条做即可。注意如何无法建立映射,则新分配的页需要释放掉。
\begin{lstlisting}[language=C]
static int
sys_page_alloc(envid_t envid, void *va, int perm)
{
	struct Env * env;
	int r = envid2env(envid, &env, 1);
	if (r < 0) return -E_BAD_ENV;

	if ((uint32_t)va >= UTOP || ROUNDUP(va, PGSIZE) != va) return -E_INVAL;
	
	if (!((perm & PTE_U) && (perm & PTE_P) && (perm & (~PTE_SYSCALL))==0)) return -E_INVAL;

	struct PageInfo * pg = page_alloc(ALLOC_ZERO);
	if (pg == NULL) return -E_NO_MEM;
	if (page_insert(env->env_pgdir, pg, va, perm) < 0) {
		// page_insert fails, should free the page you allocated!  
		page_free(pg);
		return -E_NO_MEM;
	}
	return 0;
}
\end{lstlisting}
\end{subsubsection}

\begin{subsubsection}{sys\_page\_map}
\par
按照给定的进程,和虚拟地址,拷贝其映射至另一个进程的给定的虚拟地址上,按照注释一条一条做即可。
\begin{lstlisting}[language=C]
static int
sys_page_map(envid_t srcenvid, void *srcva,
	     envid_t dstenvid, void *dstva, int perm)
{
	struct Env * dstenv, * srcenv;
	int r = envid2env(dstenvid, &dstenv, 1);
	if (r < 0) return -E_BAD_ENV;
	r = envid2env(srcenvid, &srcenv, 1);
	if (r < 0) return -E_BAD_ENV;

	if ((uint32_t)srcva >= UTOP || ROUNDUP(srcva, PGSIZE) != srcva) return -E_INVAL;
	if ((uint32_t)dstva >= UTOP || ROUNDUP(dstva, PGSIZE) != dstva) return -E_INVAL;


	// struct PageInfo * page_lookup(pde_t *pgdir, void *va, pte_t **pte_store)
	struct PageInfo * pg;
	pte_t * pte;
	pg = page_lookup(srcenv->env_pgdir, srcva, &pte);
	if (pg == NULL) return -E_INVAL;		

	if (!((perm & PTE_U) && (perm & PTE_P) && (perm & (~PTE_SYSCALL))==0)) return -E_INVAL;
	
	if ((perm & PTE_W) && ((*pte) & PTE_W) == 0) return -E_INVAL;

	// int page_insert(pde_t *pgdir, struct PageInfo *pp, void *va, int perm)
	if (page_insert(dstenv->env_pgdir, pg, dstva, perm) < 0) return -E_NO_MEM;

	return 0;
}
\end{lstlisting}
\end{subsubsection}

\begin{subsubsection}{sys\_page\_unmap}
\par
按照注释一条一条做即可。
\begin{lstlisting}[language=C]
static int
sys_page_unmap(envid_t envid, void *va)
{
	struct Env * env;
	int r = envid2env(envid, &env, 1);
	if (r < 0) return -E_BAD_ENV;

	if ((uint32_t)va >= UTOP || ROUNDUP(va, PGSIZE) != va) return -E_INVAL;
	
	// void page_remove(pde_t *pgdir, void *va)
	page_remove(env->env_pgdir, va);

	return 0;
}
\end{lstlisting}	
\end{subsubsection}

\par
注意到目前写的调度都是执行完一个environment之后就结束,剩余最后一个作为monitor,其余的CPU均HLT住,但是注意不能把所有CPU都HLT,这样会出现中断13。当我尝试在init.c中设置environment的数量大于CPU核的数量的时候,就会造成中断13,因为所有的CPU都HLT了。我在这里调试了很久,花了很长时间,最后才发现竟然是没考虑清楚。
\end{subsection}

\end{section}

\begin{section}{Part B: Copy-on-Write Fork}
\end{section}

\end{document}



















